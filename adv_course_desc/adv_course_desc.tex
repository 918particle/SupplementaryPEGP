\documentclass[../../../main.tex]{subfiles}
 
\begin{document}

\textbf{\textit{Computer Logic and Digital Circuit Design}}. Computer Logic and Digital Circuit Design is cross-listed as PHYS306/COSC330.  I have to be prudent about content selection in covering such a broad topic in an undergraduate setting. My first goal for the students was to impart my advanced learning focus of \textbf{strength in all phases of science}, and \textbf{to satisfy departmental goals 4-7}.  COSC330/PHYS306 is a 300-level integrated computer science course that satisfies core requirements in the following majors: ICS/Math, ICS/Physics, ICS/Economics, 3-2 Engineering/Math, and the scientific computing minor.  Such a broad course that serves a wide variety of students should touch on at least the following sub-topics:

\begin{enumerate}
\item Binary mathematics, non-decimal base systems, and boolean logic
\item Basic digital components, clocks and gates
\item Implementation of boolean algebra with digital components
\item Complex digital components
\end{enumerate}

Coverage of the above list with challenging and thought-provoking course content is how I reach my first advanced course learning focus, of \textit{mental discipline.}  Additionally, any good digital design course at a liberal arts setting must evenly cover the following phases of the field: \textit{mathematics, computer programming and modeling, hardware design and testing, and digital data analysis.}  In other words, in implementing this course I must reach my second learning focus of \textit{strength in all phases of science.}  The Supplemental Materials contain an example syllabus for this course.  The final learning focus is \textit{communication}, and we reached this learning goal through final group-projects coupled with a presentation at the end of the course.  In a similar fashion to the introductory courses, my student groups were required to submit a project proposal to me before beginning work, and were given the chance to polish their presentations with me in office hours in advance of the class presentation. \\ \hspace{0.1cm}

We began by diving into number systems and boolean logic, and the students seemed to engage with the material.  I required them to solve problems in pairs, in a lecture/laboratory format similar to calculus-based physics (compare to PHYS180 in 2019).  I assigned homeworks that had both quick math problems, and extended thought-provoking design questions.  This style was meant to reach my first learning focus of \textit{mental discipline}, and the first two goals of my second learning focus, \textit{strength in all phases of science}.  Unfortunately, we although I ordered the digital components for this course over a month in advance, the purchase orders were not followed by the vendor and we did not received the parts until halfway through the course.  \\ \hspace{0.1cm} 

This issue disrupted my curriculum and we had to focus on the first two goals of the second learning focus, theoretical problem solving and numerical modeling.  Once the parts arrived, we began to push forward with the final two learning goals: design and execution of technical experiments, and data analysis.  We completed our transistor radios while waiting for the digital parts to arrive, but the students felt that the project dragged on too long and that the hardware portion of the course was too tedious.  To remedy this, we will be ordering more digital components this coming Fall 2019 semester to prepare for the second iteration of the course (Spring 2020), and I have worked with our department staff to streamline the procurement processes\footnote{Some large electronics vendors do much beter with credit-card orders rather than traditional purchase orders.  Our department now has a department credit card and we can use it to acquire class components more efficiently.}.  A second improvement to better reach these goals in the second learning focus would be to incorporate more traditional content, with examples.  This is a change I made to my advanced course approach with COSC390, Digital Signal Processing that yielded excellent results (see below).\\ \hspace{0.1cm}

The final learning focus for my advanced courses is \textit{communication}, with specific goals being that the students must submit one major written or oral assignment, and that I must provide them with the opportunity to refine and clarify their work before presentation.  In the advanced course setting, the latter goal takes on more significance than it does for introductory courses.  I helped students debug faulty code, refine presentation design, and troubleshoot broken digital circuits, all during office hours or outside of class.  I felt this was important so the students could achieve success in the final result.  I further refined this process in Digital Signal Processing by requiring the students to submit their proposals much earlier in the semester.  This in part was due to the course being in January term, and partly because I've reflected on the fact that early project proposals are better.  Although the students have been exposed to less course content earlier in the semester, they have more time to refine their idea with me before beginning the project.  This strategy yielded some sharp and interesting final presentations in COSC390 (see Supplemental Materials).  \\ \hspace{0.1cm}

\textbf{\textit{Digital Signal Processing}}.  Digital Signal Processing is listed as COSC390, and I taught it for the first time in the January term of 2019.  Similar to PHYS306/COSC330, Digital Signal Processing requires prudent selection of course material in a subject that can be very broad.  COSC390 is a 300-level integrated computer science course that satisfies core requirements in the following majors: ICS/Math, ICS/Physics, ICS/Economics, 3-2 Engineering/Math, and the scientific computing minor.  I also keep in mind \textbf{Physics Department goals 4-7} when implementing this course, even though it is technically not a physics course.  \\ \hspace{0.1cm}

My first goal for the students was to reach my advanced learning focus of \textit{mental discipline} by requiring analytic and creative thinking in class and in the problem sets.  Because this was a January term course, we met for three hours each morning for three weeks.  Homework sets were assigned each day, and kept \textit{short}, but challenging.  This is the first time I tried such a style, and was more or less forced to do it due to the format of the schedule.  The students really liked it, and performed well on the assignments.  The style ensured that the problems I assigned came straight from the lecture of that day (or occasionally one day prior).  The course content touched upon the following key areas:

\begin{enumerate}
\item Statistics and probability, complex numbers, and noise in digital systems
\item Linear time-invariant (LTI) systems and filtering
\item Various DSP applications
\begin{itemize}
\item Audio systems
\item Digital images
\item Digital circuits
\item Fourier and Laplace techniques
\end{itemize}
\end{enumerate}

The second learning focus for advanced courses is \textit{strength in all phases of science.}  COSC390 follows COSC330 conceptually, and therefore contains less hardware construction and testing, and more data analysis.  One can think of Computer Logic and Digital Circuit Design as learning the logical building blocks of digital components, and some of those components lead to the ability of a computer system or scientific instrument to \textit{sample and digitize} analog data.  Digital Signal Processing (DSP) is the subject of what follows \textit{after} the sampling and digitization.  The idea that COSC390 continued from COSC330 gave me the opportunity to focus more on the data analysis and software programming phases, and to broaden the analysis to sub-topics like financial data analysis.  Indeed, one of the student-designed projects in COSC390 was an analysis of Federal Reserve interest rate data over many decades using DSP techniques. \\ \hspace{0.1cm}

The overall trajectory of COSC390 to become the completion of COSC330.  That is, after taking COSC330, students could proceed to COSC390 and thus have a ``coast-to-coast'' understanding of the collection, transfer, and analysis of digital data.  The pedagogy and style of both courses was done similarly, however, with COSC390 I had the benefit of FPC and physics department feedback.  The three main themes of this feedback were pace control, step-by-step examples, and traditional lecture content.  The main two areas on which I chose to focus were the traditional lecture content and adding more step-by-step examples.  The results were encouraging, and I reflect on them in Sec. \ref{sec:oof2}.  In Sec. \ref{sec:oof2} I compare the results from COSC330 to those of COSC390, and demonstrate the improvement. \\ \hspace{0.1cm}

\end{document}

