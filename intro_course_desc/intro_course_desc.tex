\documentclass[../../../main.tex]{subfiles}
 
\begin{document}

\begin{table}
\centering
\begin{tabular}{| c | c | c | c | c |}
\hline \hline
Semester & Course & Credits & Students & Curriculum feature \\ \hline
Fall 2017 & PHYS135A-01 & 4.0 & 24 & Intro \\ \hline
Fall 2017 & PHYS150-01 & 4.0 & 17 & COM1/Intro \\ \hline
Spring 2018 & PHYS135B-01 & 4.0 & 18 & Intro \\ \hline
Spring 2018 & PHYS180-02 & 5.0 & 19 & COM1/Intro \\ \hline
Spring 2018 & COSC330/PHYS306 & 3.0 & 6 & Advanced \\ \hline
Fall 2018 & PHYS135A-01 & 4.0 & 24 & Intro \\ \hline
Fall 2018 & PHYS135A-02 & 4.0 & 26 & Intro \\ \hline
Jan 2019 & COSC390 & 3.0 & 8 & Advanced \\ \hline
Spring 2019 & PHYS135B-01 & 4.0 & 25 & Intro \\ \hline
Spring 2019 & PHYS180-02 & 4.0 & 9 & Intro/COM1 \\ \hline
-- & Total & 39.0 & -- & -- \\ \hline
\hline
\end{tabular}
\caption{\label{tab:courses:teaching} This table is a summary of the courses I have taught since Fall 2017.  The introductory courses carry the course numbers 135A, 135B, 150, and 180.  The advanced course PHYS306 is cross-listed as a computer science course, COSC330. The other advanced course, COSC390, is now listed as a standard computer science course in the curriculum.  This table does not include the two times I have taught PHYS396 (Physics Research), which does not count for teaching credit, but is 3.0 credits per student.  Physics 396 also goes toward \textbf{Departmental goal 2}.}
\end{table}

\textbf{\textit{Algebra-based physics (135A/B)}}. Algebra-based physics, PHYS135 A/B, is a two-semester integrated lecture/laboratory sequence that covers algebra-based physics from kinematics and Newton's Laws to electromagnetism without the mathematics of calculus\footnote{See supplemental material for example syllabi.}.  Algebra-based physics is a core requirement for many technical majors such as kinesiology (KNS) and chemistry (CHEM).  Students learn problem solving in a physics context with algebra, trigonometry, and vectors.  I employ a mixture of traditional and research-based active learning methods, and use the latest version of the OpenStax open-source textbooks, \textbf{satisfying departmental goals 1, 4, and 6}.  These methods are \textit{Peer Instruction (PI)} and \textit{Physics Education Technology (PhET)}.  I no longer use JITT modules, which I experienced to be ineffective for the students (see Sec. \ref{sec:oof}).  I attended the American Association of Physics Teachers (AAPT) Workshop in 2017 to practice the implementation of these modules \footnote{See supplemental material for details.}, and I have since modified them as per department and FPC recommendations.  My total teaching credits and number of students for this course is listed in Tab. \ref{tab:courses:teaching}.  \\ \hspace{0.1cm}

The first learning focus I identify for non-majors is \textbf{curiosity}, with the measureable goals stated in Sec. \ref{sec:teaching_phil1}.  One method to help satisfy the goal of increasing their interest in physics, I encourage an activity at the beginning of each class period in which a student will present a news or science journal article pertaining to physics, published in the previous week.  I incentivise the students to volunteer as presenters by offering extra credit, which incentivizes both their practice of oral communication of scientific ideas (\textbf{Departmental goal 7}) and allows them a chance to recover points lost on midterms \footnote{Examples of such articles presented by students are included in the supplemental materials.}. \\ \hspace{0.1cm}

A second method I use to increase student curiosity is to require the students to design and complete a physics experiment.  The OpenStax textbooks contain many workable suggestions that the students can construct.  Each student group must first collectively agree on an idea, and submit a proposal to me in the middle of the semester.  I then edit with the group the proposal and ensure they have the items they need (sometimes this requires the use of our lab equipment).  After they have begun to collect data, I invite them to office hours to coach them on the presentation of the results \footnote{Included in the supplemental materials are examples of the students' final presentations.}.  By allowing the students to choose the topic and design, I provide them the opportunity to satisfy their own curiosity.  Making this assignment an oral presentation also goes toward \textbf{Departmental goal 7}.  With the two activities of article/journal presentations, and group-designed physics labs, I touch upon all three goals for the curiosity focus for introductory students.  The data in Sec. \ref{sec:oof} show that the students are reporting an increase in their curiosity for physics at an increasing rate over time. \\ \hspace{0.1cm}

The second introductory course learning focus is \textbf{improvement of analysis skill}.  PI (Peer Instruction) modules were first developed by Eric Mazur \cite{mazur}, and tend to yield higher learning gains than traditional lecture content.  Moreover, it is often helpful to illustrate physics concepts with PheT (Physics Education Technology) simulations, or to perform laboratory activities we cannot consruct (like changing the amount of gravity present) \cite{Phet}.  These two activities form the engine by which I seek to improve the analysis skill of introductory students.  However, in alignment with recommendations from my department and FPC I have attempted to balance the use of these two modules with the inclusion of more traditional lecture content in recent semesters.  Finally, after further reflecting upon the use of JITT modules \cite{jitt}, I have decided to cut them in favor of more example problems.  The students express a desire for more concrete, step-by-step examples, and less content on powerpoint slides.  Although popular PER methods claim to yield better results than traditional content, we must listen to student concerns and ideas. \\ \hspace{0.1cm}

A typical introductory physics class in the lecture/laboratory format begins with 1-2 journal/news article presentations from the students.  After a short discussion, we begin with a warm-up example on the white board from the prior class.  We then introduce new concepts on the projector screen, followed by several examples worked out on the whiteboard by me.  Third, I engage the students with a PI module pertaining to the topic at hand.  PI modules first pose a problem \textit{conceptually}, with A-E multiple choice answers.  Our classrooms are equipped with a system that records student answers anonymously.  The students take several minute to think conceptually \textit{without specific numbers or equations}, and answer on their own.  I view the answer distribution, and if fewer than 70 percent of the class answers correctly, I ask them to discuss at their table how they obtained the answer.  \\ \hspace{0.1cm}

After 2-3 minutes, I require them to submit their answer \textit{as a group} at their table.  Students often learn best from each other, as they explain their reasoning to the table.  I circulate through the classroom at this stage, seeking out the struggling students and helping them to solve the problem.  I have found deliberately focusing my attention on the most struggling students helps me to build a relationship of trust with them, and relaxes any anxieties they have with word problems.  Spending less time with the well-prepared students appears to cost me little, while the gain of spending time with those who are struggling yields dividends.  We observe the distribution of answers (choices A-E) \textit{shift} to the correct one at the end of PI modules\footnote{See supplemental material for example PI modules.}.  If the students answer correctly before the group discussion I learn that I can move on without the need for the group discussion.  This leads to the possibility of 1-2 students being left behind (if they are not in the super-majority), so I have added the concept of WAT\footnote{e.g. ``What?'' A meme indicating confusion.}.  Usually WAT corresponds to answer E, and it allows a student who is lost notify me anonymously.  If a WAT occurs, I work another example until it disappears.  \\ \hspace{0.1cm}

The second-half of the lecture/laboratory format moves on to the laboratory activity or PheT module.  An example of the difference between traditional labs and PhET modules occurs in PHYS135B and PHYS180, which cover electromagnetism.  In these courses, we often build DC electric circuits.  If the circuit is constructable in our lab, we perform a traditional experiment in which we measure voltages and electric current to verify a principle such as Ohm's Law.  If the circuit is cannot be easily built in our lab, we simply simulate it virtually with PhET software.  Whenever possible, we first simulate the circuit to make a preduction, and then construct it to compare theory and experiment in full detail.  The PI modules, PheT modules, and traditional lecture content complete my strategy for improving the students' analysis skill, and go towards \textbf{Departmental goals 1, 4, and 6}.  \textit{The student evaluation data in Sec. \ref{sec:oof} show great progress in a broad range of measures in this category.} \\ \hspace{0.1cm}

There are several avenues to reach my third introductory course learning focus, \textbf{applications to society}.  The obvious routes are the applications in the OpenStax texts \cite{openstax1} regarding kinesiology and medicine.  I develop special PI modules and example problems around topics such as motion/work/energy in the human body, nerve cells as DC circuit simulation, and lightning/weather.  Which modules I deploy depends on the semester.  I have reflected on the fact that in more recent semesters I have been much better about learning what interests the students and including content specifically for the students in my class.  Another reason why I have dropped the JITT module is that it frees up time before class for me to add material I know a student will like\footnote{See supplemental material for an example of such a unit.}.

There are two more pathways for my third learning focus.  The first is through the student-led article presentations.  An example during the past year occurred when I had an environmental science major interested in climate change in PHYS135A/B who would find climate science articles that used concepts from class to present to the group.  This type of activity empowers the students to choose topics they value, and believe have an impact on our community.  Occasionally I give hints at articles which are of high-impact for the \textit{scientific community}, and this prompt is all most timid students need to take the next step of preparing one for class.  For extra-credit I offer term-papers asking students to explain the physics of a recent or past historical discovery.  Some brilliant examples have emerged, including the history of the first measurements of the distance between the Earth and the Sun\footnote{Included in the supplemental material.}.  These papers provide a venue for students to write at length about societal impact of physics, while also touching upon \textbf{Departmental goal 7}.  \\ \hspace{0.1cm}

\textbf{\textit{Calculus-based physics (150/180)}}. Calculus-based physics, PHYS150/PHYS180, is a two-semester sequence that covers calculus-based kinematics, mechanics, thermodynamics, and electromagnetism \footnote{See supplemental material for example syllabi.}.  As with algebra-based courses, I aim to satisfy \textbf{departmental goals 1, 4, and 6}.  I have taught one section of PHYS150 and one section of PHYS180, for a total of 36 students.  As in the algebra-based classes, I implement \textit{Peer Instruction (PI)}, \textit{Just in Time Teaching (JITT)}, and \textit{Physics Education Technology (PhET)}, and use OpenStax textbooks.  The key difference between calculus and algebra-based physics methods is the increased use of PhET simulations to visualize calculus concepts.  Because PHYS150 and PHYS180 require tools from single and multi-variable calculus, students taking those courses concurrently require PhET simulations to help visualize mathematical concepts.  Examples include operations with scalar and vector fields in electromagnetism, single-variable integrals and derivatives in kinematics, and line integral calculation of work and energy. \\ \hspace{0.1cm}

To reach the first learning focus I identify for non-majors, \textbf{basic curiosity}, I use the three research-based methods plus a few other techniques.  For example, PhET simulations allow us to visualize the electric field generated by a specific charge distribution.  I can combine the field visualization with a PI module that asks the students conceptual questions about the field, including what geometric symmetry is being displayed and why.  Symmetry is an important topic in physics, but some students might not see it from equations or diagrams.  Group projects in calculus-based physics have generally been more sophisticated.  For example, students used the 3D printer to build a Sterling engine as a study of thermodynamics.  Another group studied 2D kinematics with air-pressure rockets on the football field.  A side benefit of these presentations is that the students practice good \textit{oral communication.} \\ \hspace{0.1cm}

To reach the second of the three learning focuses, \textbf{improvement of analysis skill}, I utilize the peer instruction method (PI modules), in conjunction with a procedure I learned on the fly during my first semester.  I require the students to \textbf{leave their tables, and solve the technical or numerical problem together on the whiteboards} that cover the walls of my classrooms.  Students are able to see each other's approach, and validate it against their own group's method.  Upon returning to the tables, the groups feel more prepared and eager to solve the PI module problems that follow.  The students report in their evaluations that adding this step greatly benefited their learning, and that they felt more comfortable with the material afterwards.  The students also gain analysis experience via the process of understanding statistical and systematic measurement errors.  Relative to the algebra-based activities, the calculus-based activities require a more complete understanding of error propagation. \\ \hspace{0.1cm}

To reach my focus of \textbf{applications to society}, I begin with the prompts to applications in the OpenStax texts, creating units that are relevant for the majors in my class.  Examples have included nerve signals, solar wind, and global warming.  The JITT modules demonstrate if the students have done the reading I assign, and whether they comprehend how the physics we are learning applies to society. The term-papers asking students to explain the history of science for a given topic also serve this teaching focus.  The Nobel Prize in Physics last year was for the discovery of gravitational waves, and several students chose to write about Advanced LIGO, the experiment that recorded the famous signals that have now broadened society's understanding of general relativity.  Group presentations on self-designed science projects at the end of the course offer a chance to practice oral communication skills.  Finally, I required in PHYS180 each student to briefly summarize a scientific journal article for the class, in an attempt to further practice oral communication of science. \\ \hspace{0.1cm}

\subsubsection{Descriptions of each Module Type}

\underline{PI Modules} - Implementation of an active learning strategy involving group problem solving.
\begin{itemize}
\item PI-based modules contain conceptual, multiple-choice questions for the class about a physical system.  
\item Students respond individually with an electronic device, and the distribution of answers for choices A-D is shown on the class screen (answer E indicates the student is lost).
\item One of two actions is taken next:
\begin{enumerate}
\item If the fraction of correct answers to the conceptual question is larger than 0.7, the class proceeds.
\item If the fraction is less than 0.7, the professor initiates table discussion.
\end{enumerate}
\item Table discussions take place between 2-4 students at the same table.  The professor tells the students to \textit{attempt to convince each other they are right, and that just because they gave the same answer does not indicate correctness\footnote{The effect of adding this specific phrase has been studied and shown to benefit the utility of table discussions.}.}
\item A second poll of the class is taken, to measure the increased fraction of correct answers, or \textit{gain.}  If more than one person selects E after the second round, the material is covered again.
\end{itemize}

\underline{JITT Modules} - Modification of lecture time based on student reading the day before class.
\begin{itemize}
\item JITT activities grew out of reading quizzes in a traditionally structured course.  Through Moodle, students are sent 3-4 questions the day before class based on the assigned reading.
\item JITT questions are conceptual, and if a large portion of students are answering correctly, the material is covered more lightly.  Questions that trigger many incorrect responses becomes the focus of class time.
\item JITT-module questions are drawn from a database, and tailored to common misconceptions.
\item Students' anonymous responses are included in the lecture itself, and the class gets a chance to analyze them.
\end{itemize}

\underline{PhET Modules} - Simulation activities integrated into the textbook and laboratory/PI modules.
\begin{itemize}
\item The OpenStax textbooks for PHYS135 and PHYS150/PHYS180 have built-in HTML links to JAVA-based simulations called PhET simulations\footnote{see \url{https://phet.colorado.edu}}.
\item PhET simulations are incorporated into laboratory activities, in which simulated results of a system are compared to measurements of identical systems.
\item Systems that cannot be constructed in the lab are studied via PhET activities as well.
\item PhET simulations often augment special curricular activities pertaining to other majors, like the human body.  For example, in PHYS135B we used a PhET simulation to understand the behavior of human nerve signals.
\end{itemize}

\end{document}

