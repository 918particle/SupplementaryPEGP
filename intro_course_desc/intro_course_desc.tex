\documentclass[../../../main.tex]{subfiles}
 
\begin{document}

\begin{table}
\centering
\begin{tabular}{| c | c | c | c | c |}
\hline \hline
Semester & Course & Credits & Students & Curriculum feature \\ \hline
Fall 2017 & PHYS135A-01 & 4.0 & 24 & None \\ \hline
Fall 2017 & PHYS150-01 & 4.0 & 17 & COM1 \\ \hline
Spring 2018 & PHYS135B-01 & 4.0 & 18 & None \\ \hline
Spring 2018 & PHYS180-02 & 5.0 & 19 & COM1 \\ \hline
Spring 2018 & COSC330/PHYS306 & 3.0 & 6 & Advanced course \\ \hline
-- & Total & 20.0 & -- & -- \\ \hline
\hline
\end{tabular}
\caption{\label{tab:courses:teaching} This table is a summary of the courses I have taught since Fall 2017.  The introductory courses carry the course numbers 135A, 135B, 150, and 180.  The advanced course, PHYS306, is cross-listed as a computer science course (COSC330).}
\end{table}

\textbf{\textit{Algebra-based physics (135A/B)}}. Algebra-based physics, PHYS135 A/B, is a two-semester integrated lecture/laboratory sequence that covers algebra-based kinematics, mechanics, and electromagnetism \footnote{See supplemental material for example syllabi.}.  Algebra-based physics is a core requirement for many technical majors other than physics, such as kinesiology and chemistry.  I have taught one section of PHYS135A and one section of PHYS135B, for a total of 42 students.  I am currently teaching two sections of PHYS135A with a total of 50 students.  In addition to traditional lecture-based methods, I employ research-based physics teaching methods, and use the latest version of the OpenStax open-source textbooks, \textbf{satisfying departmental goals 1, 4, and 6}.  These methods are \textit{Peer Instruction (PI)}, \textit{Just in Time Teaching (JITT)}, and \textit{Physics Education Technology (PhET)}.  I attended the American Association of Physics Teachers (AAPT) Workshop to learn how to implement these practices \footnote{See supplemental material for details.}.  (See description of module types in the next section). \\ \hspace{0.1cm}

To reach the first learning focus I identify for non-majors, \textbf{basic curiosity}, I use the three research-based methods plus a few other techniques.  For example, laboratory activities centered on constructing DC circuits and matching them to PhET simulations are meant to arouse basic curiosity about how electronics work. Second, integrating laboratory and lecture activities is meant to satisfy curiosity by providing laboratory confirmation of results derived on the board only moments ago.  Finally, group projects prompt students to design and test their own projects \footnote{Examples of student work provided in supplemental material.}. \\ \hspace{0.1cm}

To reach the second of the three learning focuses, \textbf{improvement of analysis skill}, I utilize the peer instruction method (PI modules), which has been shown to yield higher learning gains than traditional lectures concerning theoretical physics concepts.  I strive to enhance problem solving ability through repeated conceptual exercises meant to show the students that textbook problems can be translated into equations that produce answers.  After introduction of new material in the traditional sense, I provide repeated PI-modules that prompt students to examine misconceptions and use decuctive reasoning.  Sometimes I will provide a film clip or popular science article to propose a system for examination, and the class explores facets of the system with PI modules.  For example, the text provides a model of a nerve fiber transmitting an electrical signal, or a link to a TED video explaining solar wind.  After watching the clip or examining the diagram, I post a series of PI questions on the real-world topic that the class works through together. In other cases where I can physically build the system in question, we perform laboratory measurements meant to prove efficacy of a formula we derived in the lecture portion.  The students gain analysis experience via the process of understanding statistical and systematic measurement errors. \\ \hspace{0.1cm}

To reach my focus of \textbf{applications to society}, I begin with the prompts to applications in the OpenStax texts, creating units that are relevant for the majors in my class.  Examples have included nerve signals, forces in the body, and kinesiological measurements made in group projects. The JITT modules demonstrate if the students have done the reading I assign, and whether they comprehend how the physics we are learning applies to society. For extra credit, I sometimes assign term-papers asking students to explain the physics in a chapter of a science fiction novel or film, or on the history of a special scientific discovery.  Some brilliant examples have emerged, including the history of the first measurements of the distance between the Earth and the Sun. \\ \hspace{0.1cm}

\textbf{\textit{Calculus-based physics (150/180)}}. Calculus-based physics, PHYS150/PHYS180, is a two-semester sequence that covers calculus-based kinematics, mechanics, thermodynamics, and electromagnetism \footnote{See supplemental material for example syllabi.}.  As with algebra-based courses, I aim to satisfy \textbf{departmental goals 1, 4, and 6}.  I have taught one section of PHYS150 and one section of PHYS180, for a total of 36 students.  As in the algebra-based classes, I implement \textit{Peer Instruction (PI)}, \textit{Just in Time Teaching (JITT)}, and \textit{Physics Education Technology (PhET)}, and use OpenStax textbooks.  The key difference between calculus and algebra-based physics methods is the increased use of PhET simulations to visualize calculus concepts.  Because PHYS150 and PHYS180 require tools from single and multi-variable calculus, students taking those courses concurrently require PhET simulations to help visualize mathematical concepts.  Examples include operations with scalar and vector fields in electromagnetism, single-variable integrals and derivatives in kinematics, and line integral calculation of work and energy. \\ \hspace{0.1cm}

To reach the first learning focus I identify for non-majors, \textbf{basic curiosity}, I use the three research-based methods plus a few other techniques.  For example, PhET simulations allow us to visualize the electric field generated by a specific charge distribution.  I can combine the field visualization with a PI module that asks the students conceptual questions about the field, including what geometric symmetry is being displayed and why.  Symmetry is an important topic in physics, but some students might not see it from equations or diagrams.  Group projects in calculus-based physics have generally been more sophisticated.  For example, students used the 3D printer to build a Sterling engine as a study of thermodynamics.  Another group studied 2D kinematics with air-pressure rockets on the football field.  A side benefit of these presentations is that the students practice good \textit{oral communication.} \\ \hspace{0.1cm}

To reach the second of the three learning focuses, \textbf{improvement of analysis skill}, I utilize the peer instruction method (PI modules), in conjunction with a procedure I learned on the fly during my first semester.  I require the students to \textbf{leave their tables, and solve the technical or numerical problem together on the whiteboards} that cover the walls of my classrooms.  Students are able to see each other's approach, and validate it against their own group's method.  Upon returning to the tables, the groups feel more prepared and eager to solve the PI module problems that follow.  The students report in their evaluations that adding this step greatly benefited their learning, and that they felt more comfortable with the material afterwards.  The students also gain analysis experience via the process of understanding statistical and systematic measurement errors.  Relative to the algebra-based activities, the calculus-based activities require a more complete understanding of error propagation. \\ \hspace{0.1cm}

To reach my focus of \textbf{applications to society}, I begin with the prompts to applications in the OpenStax texts, creating units that are relevant for the majors in my class.  Examples have included nerve signals, solar wind, and global warming.  The JITT modules demonstrate if the students have done the reading I assign, and whether they comprehend how the physics we are learning applies to society. The term-papers asking students to explain the history of science for a given topic also serve this teaching focus.  The Nobel Prize in Physics last year was for the discovery of gravitational waves, and several students chose to write about Advanced LIGO, the experiment that recorded the famous signals that have now broadened society's understanding of general relativity.  Group presentations on self-designed science projects at the end of the course offer a chance to practice oral communication skills.  Finally, I required in PHYS180 each student to briefly summarize a scientific journal article for the class, in an attempt to further practice oral communication of science. \\ \hspace{0.1cm}

\subsubsection{Descriptions of each Module Type}

\underline{PI Modules} - Implementation of an active learning strategy involving group problem solving.
\begin{itemize}
\item PI-based modules contain conceptual, multiple-choice questions for the class about a physical system.  
\item Students respond individually with an electronic device, and the distribution of answers for choices A-D is shown on the class screen (answer E indicates the student is lost).
\item One of two actions is taken next:
\begin{enumerate}
\item If the fraction of correct answers to the conceptual question is larger than 0.7, the class proceeds.
\item If the fraction is less than 0.7, the professor initiates table discussion.
\end{enumerate}
\item Table discussions take place between 2-4 students at the same table.  The professor tells the students to \textit{attempt to convince each other they are right, and that just because they gave the same answer does not indicate correctness\footnote{The effect of adding this specific phrase has been studied and shown to benefit the utility of table discussions.}.}
\item A second poll of the class is taken, to measure the increased fraction of correct answers, or \textit{gain.}  If more than one person selects E after the second round, the material is covered again.
\end{itemize}

\underline{JITT Modules} - Modification of lecture time based on student reading the day before class.
\begin{itemize}
\item JITT activities grew out of reading quizzes in a traditionally structured course.  Through Moodle, students are sent 3-4 questions the day before class based on the assigned reading.
\item JITT questions are conceptual, and if a large portion of students are answering correctly, the material is covered more lightly.  Questions that trigger many incorrect responses becomes the focus of class time.
\item JITT-module questions are drawn from a database, and tailored to common misconceptions.
\item Students' anonymous responses are included in the lecture itself, and the class gets a chance to analyze them.
\end{itemize}

\underline{PhET Modules} - Simulation activities integrated into the textbook and laboratory/PI modules.
\begin{itemize}
\item The OpenStax textbooks for PHYS135 and PHYS150/PHYS180 have built-in HTML links to JAVA-based simulations called PhET simulations\footnote{see \url{https://phet.colorado.edu}}.
\item PhET simulations are incorporated into laboratory activities, in which simulated results of a system are compared to measurements of identical systems.
\item Systems that cannot be constructed in the lab are studied via PhET activities as well.
\item PhET simulations often augment special curricular activities pertaining to other majors, like the human body.  For example, in PHYS135B we used a PhET simulation to understand the behavior of human nerve signals.
\end{itemize}

\end{document}

