\documentclass[../../main.tex]{subfiles}
 
\begin{document}

\epigraph{\textit{That people may know wisdom and discipline, may understand intelligent sayings; May receive instruction in wise conduct, in what is right, just and fair; That resourcefulness may be imparted to the inexperienced, knowledge and discretion to the young.} - Proverbs 1:2-5}{}

\textbf{The following is a reflection on the growth and development of my teaching practices as a professor, and is submitted upon recommendation by the Faculty Personnel Committee (FPC).}  My first two years as an assistant professor of physics were filled with valuable experiences and memories.  I have strived to make important adjustments to my classroom practices, and to follow recommendations given to me by both my department and FPC.  In putting these changes in place, I have reflected on my teaching philosophy.  Looking back at my first submitted teaching philosophy, I view it as a good place to begin, but something that should evolve as I learn to handle the style of student we find specifically at Whittier College.  In the letter submitted to FPC by my department, and in the feedback letter provided to us by FPC, three general recommendations have emerged. \\ \hspace{0.1cm}

First, the \textit{pace} of my content delivery should be slowed, in order maximize student success.  Second, I must increase the number of \textit{step-by-step example problems} in my physics classes, in order to give students new to the subject and those who are struggling something concrete to grasp each time a new concept is introduced.  Third, I need to include more \textit{traditional lecture content} in my classes, which happen to take the form of an integrated lecture/laboratory format.  Traditional lecture content is a term used in physics education research (PER) to refer to the classical teaching style in which the equation corresponding to a new concept is first introduced or derived on the board, then solved in several examples and displayed in graphical form.  I'm happy to report that I have strived diligently to put these changes in place, and it has in fact yielded promising results in my classes as evidenced by my increased student evaluation numbers. \\ \hspace{0.1cm}

All of the introductory physics courses are taught in an integrated lecture/laboratory format.  I describe the lecture/laboratory format in Sec. \ref{sec:teaching_phil1} below.  I have learned in my first two years that although we do use that format by default in our department, it does not \textit{preclude} giving the students traditional content.  In fact, what works best (we have found) is a healthy mixture of the two.  The result is a style of physics course which begins new concepts traditionally, and then branches into laboratory activities and research-based lecture content when the students are ready.  \textit{Researched-based content} is a term in PER that refers to non-traditional physics teaching modules subjected to controlled research, and that have been shown to improve student learning beyond traditional content.  I described three such modules in my first PEGP: Peer-instruction (PI) \cite{mazur2013peer}, Just-in-Time Teaching (JITT) \cite{jitt}, and Physics Education Technology (PhET) \cite{phet}.  I reflect in detail in Sec. \ref{sec:teaching_phil1} on which modules tend to work best, which do not, and why. \\ \hspace{0.1cm}

\textbf{Teaching physics is about growth}.  Regardless of the physics teaching methods chosen, the student should leave the encounter \textit{enlightened}, with an increased understanding of the physics concepts.  The success of the encounter is measured by the varying degree to which the student can retain, understand, apply, and reflect upon the concepts.  The goal of the physics professor is to serve the students by formulating the concepts of physics into specific equations, testable by experimentation, and to create a problem-solving environment in which the students gain the ability to master the equations through problem-solving.  The student usually encounters failure, then the ability to find the correct solutions to specific example problems.  Finally, the professor leads the students to mastery by showing them that the concepts they have gained may be applied \textit{in general} to a broad range of problems. Each stage must be accompanied by careful laboratory experiments to verify the equations. \\ \hspace{0.1cm}

All physics begins with defining the concept of a ``system'' about which we can make measurements.  Beginning at this common place, we define the concepts of the displacement, mass, and electric charge of a system, as well as the passage of or change in time with regard to a system.  What follows is the subject of \textit{classical physics}: a description of the motions, forces and energies that govern all systems.  With the addition of concepts like temperature and heat, \textit{thermodynamics} may be added to classical physics.  Students who do not major in physics usually encounter classical physics and thermodynamics.  Physics majors progress to \textit{modern physics}, which adds the subjects of relativity and quantum mechanics to the toolkit\footnote{Students satisfying liberal arts requirements via specialty courses do experience non-classical physics qualitatively.}.  Thus, physics professors often make the distinction between \textit{physics majors} and \textit{non-majors}, who encounter different types of material.  The bulk of PER is done in the context of serving students who are non-majors, and thus the named modules (PI, JITT, and PhET) are usually applied to introductory courses.  Physics majors usually experience more traditional lecture content once they progress to upper division courses.  However, I have learned that there are exceptions to this idea that are relevant for our students at Whittier College (see Sec \ref{sec:teaching_phil1}). \\ \hspace{0.1cm}

Analogous to physics learning, learning to become a great physics instructor involves solving the basic problem of imparting simple concepts to the students and building upon their success.  The instructor must be able to generalize the teaching modules to lead students to more advanced topics, building the system of classical physics in their minds.  At each phase, the instructor must be able to guide experimentation to convince the students of the verifiable truth of the equations, while at the same time demonstrating how these equations are used to solve problems.  Upon examining my teaching practices, I have found the correct ``solution'' for our classical and introductory physics courses to be keeping the pace of the modules under control, including more concrete examples, and providing more traditional lecture content.  I have generalized these modifications to the second semesters of introductory courses that are part of two-semester sequences (when the material becomes more challenging), as well as advanced physics and computer science classes.  It was rewarding to see positive reviews in all of these courses, but I was especially pleased to see these changes pay off at the introductory course level.  Non-majors regularly tell us that they do not want to have to take these courses (see Sec. \ref{sec:oof}), so to hear the students report that these courses increased their interest in the subject matter was worth the hard work and sacrifice I have exerted in improving these courses.

\subsubsection{Instruction of Students in Introductory Courses}

\label{sec:teaching_phil1}

Physics students at Whittier College are first categorized as \textit{non-majors} or \textit{physics majors}.  Non-majors encounter physics for two semesters in either \textit{calculus-based} or \textit{algebra-based} courses.  Classical physics at the undergraduate introductory level is built upon single-variable calculus, with some multi-variable or vector calculus introduced in the second semester.  However, students who have not taken calculus can still learn the portions of the subject based on algebra and trigonometry.  Thus, \textit{non-major} students usually take the \textit{algebra-based} versions, and \textit{physics majors} and students who have chosen another technical degree take the \textit{calculus based} versions.  \\ \hspace{0.1cm}

Three focuses are relevant for teaching at the introductory level, for students concurrently taking calculus or who do not plan to take calculus:
\begin{enumerate}
\item \textbf{Curiosity}.  I regularly give colloquia at universities, seminars in physics departments, and public lectures to children, families, and astronomical societies.  Experiencing people's curiosity is necessary to become a great physics professor.  I have continued this practice as Whittier professor.  For example, I have given lectures at Los Nietos Middle School and colloquia here at Whittier College, and invited speakers from UC Irvine to give colloquia as well.  Good teaching for non-majors should \textit{entice student curiosity}.  My introductory courses include specific activities designed to entice student curiosity.  For example, I have students present science articles to the class, and give presentations on home-built experiments.  Within this teaching focus, I seek to achieve three specific goals:

\begin{itemize}
\item Measurably increase the interest of the students in physics %Q15, Q18
\item Encourage the students to satisfy their curiosity by doing self-designed experiments, through pre-designed lab activities %Labs, presentations
\item Encourage the students to present their findings to each other via presentations %Presentations
\end{itemize}

\item \textbf{Improvement of Analysis Skill}.  The scientific method relies on analytical skill.  We as physicists best serve Whittier College non-majors when we are developing their problem-solving abilities.  We apply specific modules in introductory courses (for example PI, JITT, PhET) to help train students to think analytically and hone their problem solving.  We also realize that students must learn by example, and therefore we provide healthy mixtures of traditional lecture content and step-by-step examples of problem solving.  This involves calculus as simple as converting between units of measure (i.e. kilograms to pounds) to plotting the trajectory of a particle in a vector field.  Within this teaching focus, I seek to achive three specific goals:

\begin{itemize}
\item Measurably increase the ability of the students to obtain the correct answer in word problems %Q12, Q19, Q20, learning gain in FMCE
\item Measurably increase the ability of the students to measure with precision the correct result in laboratory settings 
\end{itemize}

\item \textbf{Applications to Society}. Whittier College non-majors gain potential in technical careers if they can qualitatively explain phenomenon using physics.  In recent years, our open-source textbooks \cite{openstax1} \cite{openstax2} have included material relevant to popular majors (e.g medicine and kinesiology). I have incorporated special units centered on these applications, including human nerve systems (in PHYS135B and PHYS180).  I use group projects to allow non-majors \textit{and} non-majors to design an experiment which relates to a topic of their field and present it to the class.  I aid in the development of these projects as they progress throughout the second half of the semester.  Another tool within this learning focus is the inclusion of student-led summaries of a scientific news or journal article, which encourage scientific class discussions about the broader implications for society of the results.  Within this teaching focus, I seek to achive two specific goals:

\begin{itemize}
\item Encourage the students to present and discuss articles they find relevant or interesting due to the societal impact %Specific examples
\item Manage and aid in student-designed experiments that are presented to the class, relevant to society %Specific examples.
\end{itemize}

%Here is where I ended today.

\end{enumerate}

\subsubsection{Instruction of Students in Advanced Courses}

\label{sec:teaching_phil2}

\textit{Physics majors} are the second category of students we typically encounter.  I broaden my discussion to \textit{Mathematics and Computer Science majors} due to the specific circumstances under which I was hired.  The Departments of Mathematics and Physics at Whittier College provide the current computer science curriculum.  Our students can choose majors that combine computer science with physics, math, or economics, or join the 3-2 engineering program.  The upper-level computer science course I created is Computer Logic and Digital Circuit Design (COSC330/PHYS306).  Those who participated were physics majors, mathematics majors, and Whittier Scholars Program majors, all having some connection to computer science.  This course is under rapid development in parallel with developments in my research laboratory (see section on Scholarship below). \\ \hspace{0.1cm}

Three focuses are relevant for teaching physics, mathematics, and computer science majors at the advanced level, in addition to those above for non-majors and introductory courses:
\begin{enumerate}
\item \textbf{Mental Discipline}.  Advanced physics, math and computer science courses require discipline. and there is no substitute for grit.  I think the professor has a role in calling forth this value in the students, in two ways.  The first is delivering a rigorous curriculum.  Problem sets and exams should be difficult, requiring time and reflection.  For example, in COSC330, homeworks were assigned in two-week increments, with both mathematical repetition (to facilitate learning binary) and open-ended design questions (like designing a device that sums binary numbers).  Second, the content delivery should demonstrate expertise, but also show the students that the professor is invested in them.  Advanced classes in large universities sometimes leave the students with a blunt delivery that merely entices them to teach themselves.  The right path leaves the student \textit{motivated} to fill in gaps in their understanding, with the professor thoughtfully elevating students' understanding outside of class.  For example, in COSC330 my students and I happily debugged digital circuits in simulation software together, before building them for class presentation.

\item \textbf{Strength in all Phases of Science}. Good curriculum in these advanced topics must include the following \textit{phases} of scientific activity: theoretical problem solving, numerical modeling or simulation, experimental design and execution, and data analysis.  We may think of these phases as the actions that move the student through the scientific method.  In COSC330, an example of the incorporation of all four phases occurs in teaching the students to work with binary numbers and code.  First, the mathematics for conversion from decimal to binary is introduced along with addition and subtraction techniques, and we work example problems.  Second, we model addition and subtraction via 8-bit adders in a computer simulation.  Third, we actually build the adders, and fourth, we demonstrate that they work by analyzing the outputs.  When students gain experience in all four phases, they more firmly grasp the concept.  Students are also more likely to have a breakthrough in understanding a concept if they encounter it in multiple phases.

\item \textbf{Communication}.  Two skills that should never go overlooked in technical fields are oral and written communication.  Presentations, papers, lab reports, and summarizing peer-reviewed articles for the class are several examples of rubrics that I use in advanced courses to hone communication skills.  From personal experience, work in technical subjects would often proceed more quickly if not for the inability of group members to express themselves clearly.  When dealing with abstract concepts in engineering discussions, clear communication prevents the introduction of design flaws and the introduction of bugs in software. No matter which advanced class I am teaching, my students will write at least one report, or give one presentation.  I often allow students to write for extra credit, going beyond the scope of the course in the subject matter.  Any practice in technical writing Whittier majors receive now will benefit them down the line as they proceed to graduate school or private sector engineering careers. \footnote{See supplemental materials for examples of student presentations and writing.}
\end{enumerate}

\subsubsection{Department-Level Goals}

The Department of Physics and Astronomy has eight goals, developed as part of our 5-year assessment cycle. In the coming course descriptions, these goals will be referenced. The departmental goals are:

\begin{enumerate}
\item Develop and offer a wide range of physics courses using the most effective pedagogical methods and styles.  Such courses shall include appropriate contributions to the Liberal Education Program (currently COM1 and CON2).
\item Create research experiences for physics majors that will engage and inspire them in their discovery of physics.
\item Build a departmental community that is supportive and welcoming and that encourages students in their studies of physics.
\item Keep the physics curriculum current so that students gain the skills necessary for success in today’s scientific environment.
\item Teach students how to teach themselves. Give them the intellectual tools necessary for independent thinking and learning.
\item Train students to think ``scientifically'' i.e. critically, rigorously, quantitatively, and objectively, so that they can analyze problems and generate solutions.
\item Train students to effectively communicate scientific ideas to others.
\item Advise students about various career paths and help them along these paths.
\end{enumerate}

\end{document}

