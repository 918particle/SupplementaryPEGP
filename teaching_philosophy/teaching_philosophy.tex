\documentclass[../../main.tex]{subfiles}
 
\begin{document}

\epigraph{\textit{That people may know wisdom and discipline, may understand intelligent sayings; May receive instruction in wise conduct, in what is right, just and fair; That resourcefulness may be imparted to the inexperienced, knowledge and discretion to the young.} - Proverbs 1:2-5}{}

\textbf{The following is a reflection on the growth and development of my teaching practices as a professor, and is submitted upon recommendation by the Faculty Personnel Committee (FPC).}  My first two years as an assistant professor of physics were filled with valuable experiences and memories.  I have strived to make important adjustments to my classroom practices, and to follow recommendations given to me by both my department and FPC.  In putting these changes in place, I have reflected on my teaching philosophy.  Looking back at my first submitted teaching philosophy, I view it as a good place to begin, but something that should always evolve as I grow professionally.  In the letter submitted to FPC by my department, and in the feedback letter provided to us by FPC, three general recommendations have emerged. \\ \hspace{0.1cm}

First, the \textit{pace} of my content delivery should be slowed, in order maximize student success.  Second, I must increase the number of \textit{step-by-step example problems} in my physics classes, in order to give students new to the subject and those who are struggling something concrete to grasp before moving forward.  Third, I need to include more \textit{traditional lecture content} in my classes, which take the form of an integrated lecture/laboratory format.  Traditional lecture content is a term used in physics education research (PER) to refer to the classical teaching style in which a new equation is first introduced or derived on the board, then solved in examples and displayed in graphical form.  I have strived diligently to put these changes in place, and it has in fact yielded promising results in my classes as evidenced by my increased student evaluation numbers (see Secs. \ref{sec:oof} and \ref{sec:oof2}). \\ \hspace{0.1cm}

Introductory physics courses at Whittier are taught in an integrated lecture/laboratory format.  I describe the lecture/laboratory format in Sec. \ref{sec:teaching_phil1}.  I have learned in my first two years that although our department uses this format by default, it does not \textit{preclude} using traditional style.  In fact, what works best (we have found) is a healthy mixture of the two.  When I attend the classes of my colleagues in the physics department, this mixture is also what I observe.  We must begin new concepts traditionally, and then branch into laboratory activities and research-based lecture content when the students are ready.  \textit{Researched-based content} is a term in PER that refers physics teaching modules subjected to controlled research, and that have been shown to boost student learning beyond traditional content.  I described three such modules in my first PEGP: Peer-instruction (PI) \cite{mazur2013peer}, Just-in-Time Teaching (JITT) \cite{jitt}, and Physics Education Technology (PhET) \cite{phet}.  I reflect in Sec. \ref{sec:teaching_phil1} on which modules tend to work best, which do not, and why. \\ \hspace{0.1cm}

\textbf{Teaching physics is about growth}.  Regardless of the physics teaching methods chosen, the student should leave the encounter with an improved understanding of the physics concepts.  Success is measured by the varying degree to which the student can retain, understand, and apply the concepts.  The goal of the professor is to formulate the concepts of physics into specific equations, testable by experimentation, and to cause the students to to master the equations through problem-solving.  The student usually encounters failure, then the ability to solve specific example problems.  Finally, the professor leads the students to mastery by showing them that the concepts may be applied \textit{in general} to broad classes of problems. Each stage must be accompanied by careful laboratory experiments to verify the equations. \\ \hspace{0.1cm}

All physics subjects begin by defining a ``system,'' with measurable properties.  We define measurable quantities such as displacement, velocity, acceleration, time duration, mass, and the electric charge of a system.  What follows is the subject of \textit{classical physics}: a description of the motions, forces and energies that govern all systems.  With the addition of concepts like temperature and heat, \textit{thermodynamics} may be added to classical physics.  Students who do not major in physics usually encounter only classical physics and thermodynamics.  Physics \textit{majors} progress to \textit{modern physics}, which adds the subjects of relativity and quantum mechanics to the toolkit\footnote{Students satisfying liberal arts requirements via specialty courses do experience non-classical physics qualitatively.}.  Physics professors often distinguish between \textit{physics majors} and \textit{non-majors}, who encounter different types of material.  The bulk of PER is done in the context of serving students who are non-majors, and thus the named modules (PI, JITT, and PhET) are usually applied to introductory courses.  Upper division courses are usually taught in traditional style.  \\ \hspace{0.1cm}

Analogous to learning physics, learning to become a great physics instructor involves solving the basic problem of imparting simple concepts to the students and building upon their success.  The instructor must be able to generalize the teaching modules to lead students to more advanced topics, building the system of classical physics in their minds.  At each phase, the instructor must be able to guide laboratory experimentation, while at the same time demonstrating how the physics formulas are derived and used to solve problems.  Upon examining my teaching practices, I have found the correct ``solution'' for our classical and introductory physics courses to be keeping the pace of the modules under control, including more concrete examples, and increasing the proportion of traditional lecture content.  \\ \hspace{0.1cm}

I have built upon modifications made in the introductory courses in my advanced physics and computer science classes.  It was rewarding to see positive reviews in my new Digital Signal Processing (COSC390) course, taught in January 2019.  Almost every student in that course wrote in their evaluations that the course should be made into a full-semester length course because they liked it so much.  I was especially pleased to see the course changes that followed from the FPC and department feedback pay off at the introductory course level in PHYS135A/B and PHYS180.  Non-majors regularly report that they do not want to take the algebra-based physics courses (see Sec. \ref{sec:oof}), so to hear the students report that these courses increased their interest in physics was worth the all hard work I invested, and a great joy.

\subsubsection{Instruction of Students in Introductory Courses}

\label{sec:teaching_phil1}

Physics students at Whittier College are categorized as \textit{non-majors} or \textit{physics majors}.  Non-majors encounter physics for two semesters in either \textit{calculus-based} or \textit{algebra-based} courses.  Classical physics at the undergraduate introductory level is built upon single-variable calculus, with some multi-variable or vector calculus introduced in the second semester.  However, students who have not taken calculus can still learn using tools from algebra and trigonometry.  Thus, \textit{non-major} students usually take PHYS135A/B, and \textit{physics majors} and related majors take PHYS150/180.  \\ \hspace{0.1cm}

Three focuses are relevant for teaching at the introductory level, especially to non-majors:
\begin{enumerate}
\item \textbf{Curiosity}.  I regularly give colloquia at universities, seminars in physics departments, and public lectures to children, families, and astronomical societies.  Experiencing people's curiosity is necessary to become a great instructor.  I have continued this practice as Whittier professor.  I have given lectures at Los Nietos Middle School and colloquia here at Whittier College, and invited speakers from UC Irvine to give colloquia as well.  I am scheduled this Spring (2020) to continue with a Family Science Night at Granada Middle School.  Good teaching for non-majors should \textit{entice curiosity}, which begins by having an encounter with students where they are in their knowledge, and asking them to think more quantitatively about their physics knowledge.  My introductory courses include specific activities designed to boost curiosity.  I begin each semester with the students estimating how many candies could fill a jar, and then asking them to think about the problem more quantitatively.  Thus, they are drawn into the mathematics after becoming more curious.  Other examples including having students present science articles to the class, and give presentations on home-built experiments.  Within this teaching focus, I seek to achieve three specific goals:

\begin{itemize}
\item Measurably increase student interest in physics as measured by questions 15 and 18 on the evaluations
\item Teach the students to satisfy curiosity through self-designed experiments and pre-designed lab activities %Labs, presentations
\item Coach the public speaking skills of the students to empower them to present results to peers %Presentations
\end{itemize}

\item \textbf{Improvement of Analysis Skill}.  The scientific method relies on analytical skill.  We as physicists best serve Whittier College introductory students, especially non-majors, when we develop their problem-solving abilities.  We apply PER modules in introductory courses to train students in their problem solving.  Students also learn by example, and therefore we provide healthy mixtures of traditional lecture content and step-by-step examples of problem solving.  This involves calculations as simple as converting between units (i.e. kilograms to pounds) to plotting the trajectory of a particle in a vector field.  Within this teaching focus, I seek to achive two specific goals:

\begin{itemize}
\item Measurably increase the ability of the students to obtain the correct answer in word problems (questions 12, 19, and 20 on the evalutations)
\item Teach the students to measure with precision the correct result in laboratory settings 
\end{itemize}

\item \textbf{Applications to Society}. Whittier College students gain potential in technical careers if they can qualitatively explain phenomenon using physics.  In recent years, our open-source textbooks \cite{openstax1} \cite{openstax2} have included material relevant to popular majors (e.g medicine and kinesiology). I have incorporated special units centered on these applications, including human muscle motion (in PHYS135A) and nerve systems (in PHYS135B and PHYS180).  I help the students design experiments which can relate to their field.  One example included KNS majors in 135B who measured bicep muscle voltages for varying amounts of lifted weight.  Another tool within this learning focus is the inclusion of student-led summaries of scientific articles, which encourage class discussions about the broader implications for society.  Within this teaching focus, I seek to achive two measurable goals:

\begin{itemize}
\item Empower the students to present and discuss articles they find relevant or interesting due to the societal impact (see Supplemental Material)
\item Manage and aid in student-designed experiments that are presented to the class, relevant to society (see Supplemental Material)
\end{itemize}

\end{enumerate}

\subsubsection{Instruction of Students in Advanced Courses}

\label{sec:teaching_phil2}

\textit{Physics majors} are the second category of students we encounter, and I broaden the category to \textit{Mathematics and Computer Science majors}, because I also teach advanced computer science courses.  In my time at Whittier College, I have created two upper-division computer science courses that are part of every engineering/computer science curriculum in schools similar to Whittier College, but were not being offered here.  The first was PHYS306/COSC330, Computer Logic and Digital Circuit Design, and COSC390, Digital Signal Processing.  For sample curricula demonstrating the widespread adoption of these courses in schools like Whittier College, see \cite{BiolaCR} \cite{LMUCR}. \\ \hspace{0.1cm}

Three focuses are relevant for teaching physics, mathematics, and computer science majors at the advanced level:
\begin{enumerate}
\item \textbf{Mental Discipline}.  Advanced physics, math and computer science courses require mental discipline.  The professor must foster this value in the students in two ways.  First, the students need a professional curriculum that requires them to think \textit{analytically} and \textit{creatively}.  Second, the professor should demonstrate \textit{expertise} and the ability to lead students by example.  For example, in COSC390 (Digital Signal Processing) I wrote example code in MATLAB to demonstrate concepts from class, and the students downloaded and modified it to suit their purposes for individual projects. These ideas may be summarized into two measurable goals:

\begin{itemize}
\item Challenge the students with course content that requires both analytic and creative thinking (questions 11 and 20 from the evaluation)
\item Provide the students with technical expertise and guidance (questions 12 and 19 from the evaluation)
\end{itemize}

\item \textbf{Strength in all Phases of Science}. Advanced course curriculum in physics, math, and computer science must include the following \textit{phases} of scientific activity: abstract problem solving, numerical modeling/prediction, experimental design and execution, and data analysis. I therefore have four specific goals in this area, corresponding to the four areas:

\begin{itemize}
\item Measurably strengthen the abstract problem solving of the students (question 14 from evaluation)
\item Expose students to numerical modeling with computer code  %Examples, GitHub of DSP
\item Assist the students with the design and execution of technical projects %Examples: presentations from COSC330/390
\item Strengthen the data analysis abilities of the students %Examples: presentations from COSC330/390
\end{itemize}

\item \textbf{Communication}.  A critical skill in technical fields is oral and written communication.  Whittier College graduates in the fields of physics, mathematics, and computer science should be able to communicate technical ideas to their peers.  Clear communication in engineering and scientific research contexts prevents design flaws and misconceptions.  I require the students to submit a longer paper and/or presentation in each advanced course, with the goal of improvement of their communication of technical ideas.  To this end, I set two concrete goals:

\begin{itemize}
\item Require the students to submit at least one major written or oral assignment %Examples
\item Provide students the opportunity to refine the work in office hours before submission %Show the rubric which states they must show me the proposal
\end{itemize}

\end{enumerate}

\subsubsection{Department-Level Goals}

The Department of Physics and Astronomy has eight goals, developed as part of our 5-year assessment cycle. In the coming course descriptions, these goals will be referenced.

\begin{enumerate}
\item Develop and offer a wide range of physics courses using the most effective pedagogical methods and styles.  Such courses shall include appropriate contributions to the Liberal Education Program (currently COM1 and CON2).
\item Create research experiences for physics majors that will engage and inspire them in their discovery of physics.
\item Build a departmental community that is supportive and welcoming and that encourages students in their studies of physics.
\item Keep the physics curriculum current so that students gain the skills necessary for success in today’s scientific environment.
\item Teach students how to teach themselves. Give them the intellectual tools necessary for independent thinking and learning.
\item Train students to think ``scientifically'' i.e. critically, rigorously, quantitatively, and objectively, so that they can analyze problems and generate solutions.
\item Train students to effectively communicate scientific ideas to others.
\item Advise students about various career paths and help them along these paths.
\end{enumerate}

\end{document}

