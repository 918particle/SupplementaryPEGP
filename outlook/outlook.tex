\documentclass[../../main.tex]{subfiles}
 
\begin{document}

After reflecting on my teaching, I am optimistic in my outlook.  Teaching a student body that contains diverse skill levels has proved challenging, but many people have helped me to refine and improve as an instructor.  I specifically want to thank Drs. Lagan, Piner, and Zorba for their constructive comments and encouragement.  Each member of my department has shared experience, and has attended my lectures.  I have also enjoyed creating and giving colloquiums for my department, and I'm grateful for their interest in my research.  I'm also grateful for my wonderful research students John-Paul G\'omez-Reed, Nicolas Clarizio, Cassady Smith, and Nicolas Bakken-French.  Their enthusiasm and energy continue to inspire me, and it's been my privilege to shepherd them through fellowships with me, and to perform my research with them. I am also especially grateful to FPC for motivating me and sharing with candor ideas for improving my teaching.  I have taken these suggestions to heart and will always remain open to suggestions.  \\ \hspace{0.1cm}

My teaching is now informed by new mentoring and committee experience.  In the Fall 2019 semester, I have become a freshman adviser, and so far it has been a joyful experience to welcome new students to our community.  I am learning how to guide them outside the classroom in their future plans.  I teach them in both my new CON2 liberal arts course and in physics, and they have already begun to see the connections.  Further, for the 2018-2019 academic year I served the Enrollment and Student Affairs committee and served the sub-committee on student admissions data analysis.  I was (possibly) the first person to discover the bi-modal nature of our aid gap distribution, and revealed to the committee how the vast majority of our unsuccessful students do in fact come from the side of the aid gap distribution representing \textit{debt} rather than \textit{surplus} tuition.  Service has informed my teaching in two ways.  I see the financial struggle some students undertake just to keep pace in college, and have gained insight into varying levels of preparation with which our students arrive.  \\ \hspace{0.1cm}

Looking forward to the future semesters, I plan to offer new courses.  Since I have been a professor here, I have created three new courses: \textit{Computer Logic and Digital Circuit Design} (COSC330/PHYS306), \textit{Digital Signal Processing} (COSC390), and \textit{Safe Return Doubtful: History and Current Status of Modern Science in Antarctica} (INTD255, a CON2).  INTD255 is my first contribution to the liberal arts curriculum, and it explores polar exploration from a scientific and historical standpoint.  The centerpiece is the discovery of the South Pole, and the subtext of the course is that the students must undergo a journey of \textit{self}-discovery in order to plan their futures.  Additionally, Prof. Michelle Chihara was kind enough to suggest books that encapsulate the fascinating tradition of polar literature, which has enhanced the course.  The students, including my advisees, seem to be enjoying the course.  \\ \hspace{0.1cm}

My January 2019 course, Digital Signal Processing, was so well received that the students all asked for me to morph it into a semester-long course.  This is to enable further presentation of applications of DSP relevant to music and image processing, analytics, machine learning, and economics.  I have already had discussions with the chair of the Math Department, who has given me the green light.  Our department planning includes a semester that is open periodically for me to teach such a course, and the demand seems to be large.  This course will therefore connect in interesting ways to subjects within other fields, such as music and economics.  Finally, I do want to reassure the FPC that I am still interested in offering another history of science course: History of Science in Latin America.  I have given myself time for my Spanish to improve, and am in the process of developing the course content for the future. \\ \hspace{0.1cm}

Once again, I am grateful to my department and to the Faculty Personnel Committee for the feedback and guidance.  I look forward to sharing all the exciting developments in my recent scholarship with the committee in my full report in Fall 2020. \\ \hspace{0.1cm}

Respectfully submitted, \\
Dr. Jordan Hanson\\
Assistant Professor of Physics\\
Department of Physics and Astronomy\\
Whittier College

\end{document}

