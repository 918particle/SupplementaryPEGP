\documentclass[../../main.tex]{subfiles}
 
\begin{document}

After reflecting on my teaching, I am optimistic in my outlook.  Teaching students with diverse skill levels is challenging, but many people have helped me to improve as an instructor.  I want to thank Drs. Lagan, Piner, and Zorba for their constructive comments and encouragement.  Each member of my department has shared experience and attended my lectures.  I have also enjoyed giving colloquia for my department, and I'm grateful for their interest in my research.  I'm also grateful for my wonderful research assistants John-Paul G\'omez-Reed, Nicolas Clarizio, Cassady Smith, and Nicolas Bakken-French.  Their enthusiasm and energy inspire me, and it has been my privilege to shepherd them through Keck and Ondrasik-Groce fellowships. I am also especially grateful to FPC for motivating me and sharing with candor ideas for improving my teaching.  I have taken these suggestions to heart and will always remain open to suggestions.  \\ \hspace{0.1cm}

My teaching is now informed by new mentoring and committee experience.  In the Fall 2019 semester, I have become a freshman adviser, and so far it has been a joyful experience to welcome new students to our community.  I am learning how to guide them outside the classroom in their future plans.  I teach them in both my new CON2 liberal arts course and in physics, and they have already begun to see the connections.  Further, for the 2018-2019 academic year I served the Enrollment and Student Affairs committee and served the sub-committee on student admissions data analysis.  I was (possibly) the first person to discover the bi-modal nature of our aid gap distribution, and revealed how the vast majority of students who leave Whittier before their second year come from the side of the aid gap distribution representing \textit{debt} rather than \textit{surplus} tuition.  Now I can understand the financial struggle some students undertake just to keep pace in college, and have gained insight into the preparation with which our students arrive.  \\ \hspace{0.1cm}

Another form of service I undertook in the 2018-2019 year was with the Artemis Program, in conjunction with the Center for Engagement with Communities.  The Artemis Program is a community-based program designed to bring young ladies from local high-schools to perform a research project with professors at Whittier.  During the 2018-2019 academic year, a Whittier student named Samantha Ruiz helped me to organize a research project and recruit four young ladies from high schools in our community.  I taught them how to write code in python, and together we ran a research project that used python code to gather data on how quickly and efficiently people solve physics problems.  Finally, once we had gathered data, we revealed that some of their high-school peers solved certain types of problems faster than others.  These bright young students are now off to present their poster of our work at the Southern California Conferences for Undergraduate Research (\url{https://www.sccur.org/}). \\ \hspace{0.1cm}

Looking forward, I will continue to help develop the STEM and liberal arts curriculum.  Since arriving, I have created three new courses: \textit{Computer Logic and Digital Circuit Design} (COSC330/PHYS306), \textit{Digital Signal Processing} (COSC390), and \textit{Safe Return Doubtful: History and Current Status of Modern Science in Antarctica} (INTD255, a CON2).  INTD255 is my first liberal arts contribution, and it explores polar exploration from a scientific and historical standpoint.  The centerpiece is the discovery of the South Pole, and the subtext of the course is that the students must undergo a journey of \textit{self}-discovery in order to plan their futures.  Additionally, Prof. Michelle Chihara was kind enough to suggest books that encapsulate the fascinating tradition of polar literature, which has enhanced the course.  The students, which include my advisees, seem to be enjoying the course.  \\ \hspace{0.1cm}

Regarding new courses, Digital Signal Processing was so well received that the students all asked for me to morph it into a semester-long course.  This is to enable further presentation of applications of DSP relevant to music and image processing, analytics, machine learning, and economics.  I have already had discussions with the chair of the Math Department, who has given me the green light.  Our department planning includes a semester that is open periodically for me to teach such a course, and the demand seems to be large.  This course will therefore connect in interesting ways to subjects within other fields, such as music and economics.  Finally, I do want to reassure the FPC that I am still interested in offering another liberal arts course: History of Science in Latin America.  I have given myself time for my Spanish to improve, and am in the process of developing the course content for the future. \\ \hspace{0.1cm}

Once again, I am grateful to my department and to the Faculty Personnel Committee for the feedback and guidance.  I look forward to sharing all the exciting developments in my recent scholarship with the committee in my full report in Fall 2020. \\ \hspace{0.1cm}

Respectfully submitted, \\
Dr. Jordan Hanson\\
Assistant Professor of Physics\\
Department of Physics and Astronomy\\
Whittier College

\end{document}

