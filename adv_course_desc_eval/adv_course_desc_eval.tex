\documentclass[../../main.tex]{subfiles}

What follows is an analysis of student evaluations and reflections on my teaching in the 2018-2019 academic year for the advanced courses.  Similar to the introductory course patterns, the analysis reflects the modifications and improvements made following recommendations from my department and FPC.  I also strived to improve COSC390 based on my experience in COSC330 by focusing on building productive relationships with the students.  The reflections focus on qualitative experiences from class that show I am reaching the specific goals within the learning focuses defined in Sec. \ref{sec:teaching_phil1}.  \\ \hspace{0.1cm}
 
\begin{document}
\label{sec:oof2}

\begin{table}
\small
\centering
\begin{tabular}{| c | c | c |}
\hline \hline
Question & COSC390 $N$ & COSC390 result \\ \hline
10 & 8 & $4.75\pm 0.71$ \\ \hline
11 & 8 & $4.75\pm 0.46$ \\ \hline
12 & 8 & $4.75\pm 0.46$ \\ \hline
13 & 8 & $4.75\pm 0.46$ \\ \hline
14 & 8 & $4.75\pm 0.46$ \\ \hline
15 & 8 & $4.63\pm 0.52$ \\ \hline
16 & 8 & $4.50\pm 0.76$ \\ \hline
\hline
\end{tabular}
\quad
\begin{tabular}{| c | c | c |}
\hline \hline
Question & COSC390 $N$ & COSC390 result \\ \hline
17 & 8 & $4.75\pm 0.46$ \\ \hline
18 & 8 & $4.75\pm 0.71$ \\ \hline
19 & 8 & $4.50\pm 0.76$ \\ \hline
20 & 8 & $4.63\pm 0.74$ \\ \hline
21 & 8 & $4.88\pm 0.35$ \\ \hline
22 & 8 & $4.14\pm 0.69$ \\ \hline
23 & 8 & $4.17\pm 0.98$ \\ \hline
24 & 8 & $5.00\pm 0.00$ \\ \hline
25 & 8 & $4.75\pm 0.46$ \\ \hline
\hline
\end{tabular}
\caption{\label{tab:courses:adv_eval_3} (Left) Mean and standard deviation for Questions 10-16 on the student evaluation form, for COSC390, taught in January 2019.  These questions pertain to the \textit{course}.  (Right) Mean and standard deviation for questions 17-25 on the student evaluation form, for COSC390, taught in January 2019.  These questions pertain to the \textit{professor}.}
\end{table}

Table \ref{tab:courses:adv_eval_3} contains the student evaluation responses to Questions 10-16 (left) and Questions 17-25 (right).  \textbf{The data shows that the students were very pleased with this course}, and I am particularly proud of two specific results.  Question 16\footnote{``Overall, I would recommend this course to others.''} reflects the fact that the students approved of the course.  This was the first time I have taught DSP, and the first time it has been taught \textit{ever} at Whittier College.  This is especially interesting considering that the accelerated January term schedule carried the potential for the students to feel lost with the onrush of a lot of material.  Apparently this was not the case, and their answers to Question 14\footnote{``This course improved my understanding of the material.''} indicate that their understanding of the material \textit{increased.} \\ \hspace{0.1cm}

Table \ref{tab:courses:adv_eval_3} also shows that \textbf{the students approved of my teaching}.  Question 25\footnote{``Overall, I would recommend this professor to others.''} indicates they were pleased with my style, and Question 19\footnote{``The professor was able to explain complicated ideas.''} shows that even though the DSP material can be challenging, the students understood.  I would have liked to score higher on Question 22\footnote{``The professor encouraged meaningful class discussions.''}, but time for this was limited in January term.  Question 23 I consider linked to 22, since it pertains to differing viewpoints\footnote{``The professor was receptive to differing views.''}, but DSP is a technical topic not always ammenable to open-ended discussions in the same manner we might encounter in other courses.  I should think more carefully about connecting the DSP topics covered in COSC390 to real-world applications, and include discussion time to brainstorm how the topics might be applied in new ways. \\ \hspace{0.1cm}

I have reflected on the written responses from student evaluations from COSC390, and there are two common threads.  The most common thread is that they want this course to be made into a semester-long course, in order to cover the topics they liked in more detail.  They would have liked more deep dives into the later chapters, when we began to reach the image and audio processing in detail.  I encouraged students to take these deeper looks at those topics for their final presentations, but ultimately time constraints limited what we could do.  The second thread is that people liked the course, but wanted to slow the pace and focus more on the applications over the basics of DSP.  I agree that we should do this, and I have taken action to create a listed full-semester DSP course. \\ \hspace{0.1cm}

\begin{table}
\small
\centering
\begin{tabular}{| c | c | c | c | c |}
\hline
\hline
Question & First Time & Most Recent Time & Raw change & Standard deviations \\
\hline
10 & 3.13 $\pm$ 0.516 & 4.75 $\pm$ 0.251 & 1.62 $\pm$ 0.574 & 2.82 \\ \hline
11 & 3.71 $\pm$ 0.488 & 4.75 $\pm$ 0.163 & 1.04 $\pm$ 0.514 & 2.02 \\ \hline
12 & 3.75 $\pm$ 0.368 & 4.75 $\pm$ 0.163 & 1 $\pm$ 0.402 & 2.49 \\ \hline
13 & 3.25 $\pm$ 0.491 & 4.75 $\pm$ 0.163 & 1.5 $\pm$ 0.518 & 2.9 \\ \hline
14 & 3.63 $\pm$ 0.421 & 4.75 $\pm$ 0.163 & 1.12 $\pm$ 0.451 & 2.48 \\ \hline
15 & 3.86 $\pm$ 0.244 & 4.63 $\pm$ 0.184 & 0.77 $\pm$ 0.305 & 2.52 \\ \hline
16 & 3.29 $\pm$ 0.442 & 4.5 $\pm$ 0.269 & 1.21 $\pm$ 0.517 & 2.34 \\ \hline
17 & 3.38 $\pm$ 0.566 & 4.75 $\pm$ 0.163 & 1.37 $\pm$ 0.589 & 2.33 \\ \hline
18 & 3.5 $\pm$ 0.424 & 4.75 $\pm$ 0.251 & 1.25 $\pm$ 0.493 & 2.54 \\ \hline
19 & 3.13 $\pm$ 0.516 & 4.5 $\pm$ 0.269 & 1.37 $\pm$ 0.582 & 2.35 \\ \hline
20 & 4.25 $\pm$ 0.251 & 4.63 $\pm$ 0.262 & 0.38 $\pm$ 0.363 & 1.05 \\ \hline
21 & 3.5 $\pm$ 0.499 & 4.88 $\pm$ 0.124 & 1.38 $\pm$ 0.514 & 2.69 \\ \hline
22 & 4 $\pm$ 0.315 & 4.14 $\pm$ 0.244 & 0.14 $\pm$ 0.398 & 0.352 \\ \hline
23 & 4.25 $\pm$ 0.41 & 4.17 $\pm$ 0.346 & -0.08 $\pm$ 0.537 & -0.149 \\ \hline
24 & 4.29 $\pm$ 0.442 & 5 $\pm$ 0 & 0.71 $\pm$ 0.442 & 1.61 \\ \hline
25 & 2.88 $\pm$ 0.481 & 4.75 $\pm$ 0.163 & 1.87 $\pm$ 0.508 & 3.68 \\ \hline
\hline
\end{tabular}
\caption{\label{tab:courses:adv_shifts} Comparison of COSC330 results (mean and error in the mean) for the first time taught (first column) to the first time teaching COSC390. The raw change is given in the third column, and the change divided by the standard deviation is given in the fourth column.}
\end{table}

Table \ref{tab:courses:adv_shifts} compares the student evaluation results from Computer Logic and Digital Circuit Design (COSC330) to Digital Signal Processing (COSC390).  This is a fair comparison for the reasons given in Sec. \ref{sec:adv_desc}.  \textbf{The data show significant increases in student evaluation scores when comparing the two courses.}  Like the introductory courses, each measurement shows an increase of 2-3 standard deviations.  The only exception is Question 23, pertaining to being receptive to differing views.  It is difficult to have opinion-based discussions in both DSP and Computer Logic, because these topics are not as ammenable to such discussions as other courses.  That being said, I think there is a venue for sharing differing viewpoints regarding the \textit{applications} of the technologies we create.  I was relieved and excited to see that the increase in the mean value to Question 25 regarding recommending me as a professor has increased \textit{by almost four standard deviations}. \\ \hspace{0.1cm}

Below I include a letter my student John-Paul G\'{o}mez-Reed sent to myself and the chair of the Department of Mathematics at the conclusion of the course: \\ \hspace{0.1cm}

\textit{Hello Dr. Kronholm, \\ \vspace{0.5cm} I have recently completed the January term COSC390 and would like to say that it was an excellent course. It was incredibly informative and related many concepts that were foreign to me with concrete examples, like the math behind image processing.  Furthermore, it also provided with exposure with a new programming language: Octave\/Matlab.  I am emailing you to say that I feel that the course should be promoted from a January term course to a full semester course.  I enjoyed my time learning during Jan term, but I feel that making COSC390 a full semester course would allow the course to reach it's full potential; time constraints led to less topics being covered, like the Laplace transform.  In any case, I enjoyed my time in the course and hope that COSC390 does become a full semester course. \\ \vspace{0.5cm} Thank you for your time and have a good afternoon, \\ John Paul G\'omez-Reed}

\end{document}