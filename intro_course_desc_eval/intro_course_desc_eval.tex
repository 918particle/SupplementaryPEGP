\documentclass[../../main.tex]{subfiles}
 
\begin{document}
\label{sec:oof}

\begin{table}
\small
\centering
\begin{tabular}{| c | c | c | c | c |}
\hline
\hline
Question & First Time & Most Recent Time & Raw change & Standard deviations \\
\hline
10 & 3.76 $\pm$ 0.227 & 4.46 $\pm$ 0.159 & 0.7 $\pm$ 0.277 & 2.53 \\ \hline
11 & 4.57 $\pm$ 0.164 & 4.42 $\pm$ 0.159 & -0.15 $\pm$ 0.228 & -0.657 \\ \hline
12 & 4.29 $\pm$ 0.22 & 4.54 $\pm$ 0.159 & 0.25 $\pm$ 0.272 & 0.919 \\ \hline
13 & 3.52 $\pm$ 0.29 & 4.42 $\pm$ 0.19 & 0.9 $\pm$ 0.347 & 2.6 \\ \hline
14 & 3.48 $\pm$ 0.297 & 4.46 $\pm$ 0.169 & 0.98 $\pm$ 0.342 & 2.87 \\ \hline
15 & 3.29 $\pm$ 0.367 & 4.25 $\pm$ 0.21 & 0.96 $\pm$ 0.423 & 2.27 \\ \hline
16 & 3.19 $\pm$ 0.343 & 4.33 $\pm$ 0.188 & 1.14 $\pm$ 0.391 & 2.92 \\ \hline
17 & 4.24 $\pm$ 0.227 & 4.57 $\pm$ 0.149 & 0.33 $\pm$ 0.271 & 1.22 \\ \hline
18 & 3.52 $\pm$ 0.29 & 4.48 $\pm$ 0.174 & 0.96 $\pm$ 0.338 & 2.84 \\ \hline
19 & 3.48 $\pm$ 0.306 & 4.38 $\pm$ 0.167 & 0.9 $\pm$ 0.348 & 2.58 \\ \hline
20 & 4.24 $\pm$ 0.238 & 4.52 $\pm$ 0.174 & 0.28 $\pm$ 0.294 & 0.951 \\ \hline
21 & 4.48 $\pm$ 0.225 & 4.54 $\pm$ 0.169 & 0.06 $\pm$ 0.281 & 0.213 \\ \hline
22 & 4.1 $\pm$ 0.194 & 4.48 $\pm$ 0.202 & 0.38 $\pm$ 0.28 & 1.36 \\ \hline
23 & 3.95 $\pm$ 0.262 & 4.64 $\pm$ 0.135 & 0.69 $\pm$ 0.294 & 2.34 \\ \hline
24 & 4.67 $\pm$ 0.127 & 4.75 $\pm$ 0.108 & 0.08 $\pm$ 0.167 & 0.48 \\ \hline
25 & 3.24 $\pm$ 0.338 & 4.46 $\pm$ 0.169 & 1.22 $\pm$ 0.378 & 3.22 \\ \hline
\hline
\end{tabular}
\caption{\label{tab:courses:intro_shifts} Comparison algebra-based numbers for the first time taught (first column) to the most recent time (second column). The raw change is given in the third column, and the change divided by the standard deviation is given in the fourth column.}
\end{table}

\begin{table}
\small
\centering
\begin{tabular}{| c | c | c | c | c |}
\hline
\hline
Question & First Time & Most Recent Time & Raw change & Standard deviations \\
\hline
10 & 4.19 $\pm$ 0.207 & 5 $\pm$ 0 & 0.81 $\pm$ 0.207 & 3.9 \\ \hline
11 & 4.19 $\pm$ 0.345 & 5 $\pm$ 0 & 0.81 $\pm$ 0.345 & 2.35 \\ \hline
12 & 3.63 $\pm$ 0.327 & 5 $\pm$ 0 & 1.37 $\pm$ 0.327 & 4.18 \\ \hline
13 & 4 $\pm$ 0.275 & 5 $\pm$ 0 & 1 $\pm$ 0.275 & 3.64 \\ \hline
14 & 3.93 $\pm$ 0.333 & 5 $\pm$ 0 & 1.07 $\pm$ 0.333 & 3.22 \\ \hline
15 & 3.56 $\pm$ 0.315 & 5 $\pm$ 0 & 1.44 $\pm$ 0.315 & 4.57 \\ \hline
16 & 3.56 $\pm$ 0.315 & 5 $\pm$ 0 & 1.44 $\pm$ 0.315 & 4.57 \\ \hline
17 & 3.31 $\pm$ 0.285 & 5 $\pm$ 0 & 1.69 $\pm$ 0.285 & 5.93 \\ \hline
18 & 2.88 $\pm$ 0.34 & 5 $\pm$ 0 & 2.12 $\pm$ 0.34 & 6.24 \\ \hline
19 & 3.13 $\pm$ 0.385 & 5 $\pm$ 0 & 1.87 $\pm$ 0.385 & 4.86 \\ \hline
20 & 3.69 $\pm$ 0.312 & 5 $\pm$ 0 & 1.31 $\pm$ 0.312 & 4.19 \\ \hline
21 & 3.88 $\pm$ 0.273 & 5 $\pm$ 0 & 1.12 $\pm$ 0.273 & 4.11 \\ \hline
22 & 3.81 $\pm$ 0.333 & 4.75 $\pm$ 0.251 & 0.94 $\pm$ 0.417 & 2.26 \\ \hline
23 & 3.67 $\pm$ 0.343 & 5 $\pm$ 0 & 1.33 $\pm$ 0.343 & 3.88 \\ \hline
24 & 4.5 $\pm$ 0.157 & 5 $\pm$ 0 & 0.5 $\pm$ 0.157 & 3.17 \\ \hline
25 & 3.13 $\pm$ 0.407 & 5 $\pm$ 0 & 1.87 $\pm$ 0.407 & 4.59 \\ \hline
\hline
\end{tabular}
\caption{\label{tab:courses:intro_shifts_2} Comparison calculus-based numbers for the first time taught (first column) to the most recent time (second column). The raw change is given in the third column, and the change divided by the standard deviation is given in the fourth column.}
\end{table}

\begin{table}
\small
\centering
\begin{tabular}{| c | c | c | c | c | c | c |}
\hline \hline
Question & 135A $N$ & 135A Mean & 135A Std. dev. & 135B $N$ & 135B Mean & 135B Std. dev. \\ \hline
10 & 21 & 3.76 & 1.04 & 18 & 3.72 & 0.96 \\ \hline
11 & 21 & 4.57 & 0.75 & 18 & 4.78 & 0.43 \\ \hline
12 & 21 & 4.29 & 1.01 & 18 & 3.78 & 1.00 \\ \hline
13 & 21 & 3.52 & 1.33 & 18 & 3.33 & 1.53 \\ \hline
14 & 21 & 3.48 & 1.36 & 18 & 2.72 & 1.32 \\ \hline
15 & 21 & 3.29 & 1.68 & 18 & 2.28 & 1.53 \\ \hline
16 & 21 & 3.19 & 1.57 & 18 & 2.94 & 1.30 \\ \hline
\hline
\end{tabular}
\caption{\label{tab:courses:intro_eval_1} Summary of questions 10-16 on the student evaluation form, for PHYS135A/B taught in Fall 2017 and Spring 2018.  These questions pertain to the \textit{course}.}
\end{table}

\begin{table}
\small
\centering
\begin{tabular}{| c | c | c | c | c | c | c |}
\hline \hline
Question & 135A $N$ & 135A Mean & 135A Std. dev. & 135B $N$ & 135B Mean & 135B Std. dev. \\ \hline
17 & 21 & 4.24 & 1.04 & 18 & 3.67 & 1.03 \\ \hline
18 & 21 & 3.52 & 1.33 & 18 & 3.11 & 1.57 \\ \hline
19 & 21 & 3.48 & 1.40 & 18 & 2.89 & 1.29 \\ \hline
20 & 21 & 4.24 & 1.09 & 18 & 4.06 & 1.25 \\ \hline
21 & 21 & 4.48 & 1.03 & 18 & 3.78 & 1.17 \\ \hline
22 & 21 & 4.10 & 0.89 & 18 & 3.88 & 1.02 \\ \hline
23 & 21 & 3.95 & 1.20 & 18 & 3.53 & 1.33 \\ \hline
24 & 21 & 4.67 & 0.58 & 18 & 4.24 & 0.97 \\ \hline
25 & 21 & 3.24 & 1.55 & 18 & 3.12 & 1.36 \\ \hline
\hline
\end{tabular}
\caption{\label{tab:courses:intro_eval_2} Summary of questions 17-25 on the student evaluation form, for PHYS135A/B taught in Fall 2017 and Spring 2018.  These questions pertain to the \textit{professor}.}
\end{table}

\begin{table}
\small
\centering
\begin{tabular}{| c | c | c | c | c | c | c |}
\hline \hline
Question & 150 $N$ & 150 Mean & 150 Std. dev. & 180 $N$ & 180 Mean & 180 Std. dev. \\ \hline
10 & 16 & 4.19 & 0.83 & 18 & 4.00 & 0.91 \\ \hline
11 & 16 & 4.19 & 1.38 & 18 & 4.67 & 0.49 \\ \hline
12 & 16 & 3.63 & 1.31 & 18 & 4.06 & 0.94 \\ \hline
13 & 16 & 4.00 & 1.10 & 18 & 4.00 & 0.97 \\ \hline
14 & 16 & 3.93 & 1.33 & 18 & 3.89 & 0.90 \\ \hline
15 & 16 & 3.56 & 1.26 & 18 & 3.67 & 1.03 \\ \hline
16 & 16 & 3.56 & 1.26 & 18 & 3.83 & 0.86 \\ \hline
\hline
\end{tabular}
\caption{\label{tab:courses:intro_eval_3} Summary of questions 10-16 on the student evaluation form, for PHYS150 taught in Fall 2017, and PHYS1809 taught in Spring 2018.  These questions pertain to the \textit{course}.}
\end{table}

\begin{table}
\small
\centering
\begin{tabular}{| c | c | c | c | c | c | c |}
\hline \hline
Question & 150 $N$ & 150 Mean & 150 Std. dev. & 180 $N$ & 180 Mean & 180 Std. dev. \\ \hline
17 & 16 & 3.31 & 1.14 & 18 & 3.44 & 1.15 \\ \hline
18 & 16 & 2.88 & 1.36 & 18 & 3.39 & 1.14 \\ \hline
19 & 16 & 3.13 & 1.54 & 18 & 3.83 & 1.04 \\ \hline
20 & 16 & 3.69 & 1.25 & 18 & 4.22 & 0.65 \\ \hline
21 & 16 & 3.88 & 1.09 & 18 & 4.11 & 0.96 \\ \hline
22 & 16 & 3.81 & 1.33 & 18 & 4.44 & 0.70 \\ \hline
23 & 16 & 3.67 & 1.37 & 18 & 4.33 & 0.77 \\ \hline
24 & 16 & 4.50 & 0.63 & 18 & 4.56 & 0.51 \\ \hline
25 & 16 & 3.13 & 1.63 & 18 & 3.61 & 1.04 \\ \hline
\hline
\end{tabular}
\caption{\label{tab:courses:intro_eval_4} Summary of questions 17-25 on the student evaluation form, for PHYS150 taught in Fall 2017, and PHYS180 taught in Spring 2018.  These questions pertain to the \textit{professor}.}
\end{table}

Tables \ref{tab:courses:intro_eval_1} and \ref{tab:courses:intro_eval_2} show the results of the \textit{algebra-based} introductory physics courses taught in the 2017-2018 academic year.  Tables \ref{tab:courses:intro_eval_3} and \ref{tab:courses:intro_eval_4} show the results of the \textit{calculus-based} introductory physics courses taught in the 2017-2018 academic year.  The results show an interesting correlation that reveals a potential strategy for continual improvement of my teaching in these courses. \\ \hspace{0.1cm}

First, there are areas that need improvement.  Questions 14-16 and 19 read as follows: ``This course improved my understanding of the material,'' ``This course increased my interest in the subject matter,'' ``Overall, I would recommend this course to others,'' and ``The professor was able to explain complicated ideas,'' respectively.  For \textit{algebra-based physics}, there are no prerequisites, but students are required to solve problems involving algebraic equations, graphical analysis, and concepts like vectors and vector fields.  It is not surprising that students struggle if they're encountering these concepts for the first time.  I have been trained to teach much faster than the students who disapproved expected.  \\ \hspace{0.1cm}

In the students' written responses, the most common remark was that the pace of the course was too fast, and that they desired more traditional lecture time with explicit examples given.  Some students remarked that the portions of the lecture on the projector (e.g. PI-modules, JITT-modules) were not as helpful as the traditional style with examples.  In Fall 2017, Professor Serkan Zorba and I both taught PHYS135A, and in Spring 2018 we both taught PHYS135B, meaning that some students had to switch professors.  Students who switched were newly stressed by the increased pace, in addition to the students I had from the Fall.  Some students even met with me in office hourse to brainstorm ways in which we could move more slowly, but still cover the necessary book chapters.  These meetings were helpful, and I learned that the difference in my expectation of student physics preparation and their actual preparedness was wide.  Now that I fully understand the problem, which is more pronounced for 135B, I can begin to solve it.  \\ \hspace{0.1cm}

PHYS135B introduces the concept of vector \textit{fields}, necessary for understanding electromagnetic fields.  This new concept further added difficulty for students encountering vectors for the first time\footnote{A single vector describes, for example, the velocity of a single leaf blown in a direction by the wind.  A vector field, on the other hand, describes the wind velocity at all points in space.}.  The research-based methods such as PI-modules rely on groups of students teaching each other.  If a whole group is struggling, they don't gain the benefit of the one student who understands the problem to show them how to solve it.  Thus, the average scores on questions 14, 16 and 19 showed slight decreases (less than one standard deviation).  Responses to Question 11, ``This course was academically challenging'' showed a slight increase, which is evidence that the students found the second semester more difficult than the first.  I assess this further in the Appendix.  Figure \ref{fig:ag_data} of the Appendix shows that when I score high, there is broad consensus among the students that I am performing well.  However, when the \textit{average} scores are lower, there is less consenses (a larger variance in the data).  \\ \hspace{0.1cm}

I did not encounter many written responses in which students expressed strong feelings about question 15, ``This course increased my interest in the subject matter.''  Most students taking PHYS135A/B are fulfilling a requirement for their major (a two-thirds majority are KNS majors), and typically do not express a strong desire to take the course\footnote{Question 9: ``I had a strong desire to take this course.'' PHYS135A/B students reported $3.24\pm 1.64$ and $3.00\pm 1.71$, respectively.}.  Nevertheless, I have attempted to add content that appeals to KNS majors.  For example, when we reach concepts in PHYS135A pertaining to biomechanics, such as torque, I give specific example problems and final project assignments that relate in some way to torque in the human body.  Another example pertains to PHYS135B, which addresses electric current.  When we reach the topic of current, the class solves problems specific to the electrical currents in the human nervous system\footnote{In the supplemental materials I include a student's final presentation on the electrical nerve activity of a bicep under torque.}.  For more information on KNS-relevant material in PHYS135A/B, see Sec. \ref{sec:future}. \\ \hspace{0.1cm}

In \textit{calculus-based physics}, the story was different.  There are many areas in which the courses and my teaching scored well.  \textbf{I am especially proud of the fact that Q25 (``Overall, I would recommend this professor to others'') jumped in 2018 relative to 2017 for the calculus-based courses.  In fact, my teaching scores improved in almost every category in calculus-based physics in going from Fall 2017 to Spring 2018.}  Further, students in both sections believed that the courses were rigorous and challenging, while still giving me increasing marks in all categories.  Some students appreciated the PI modules, PhET simulations, and JITT exercises.  This is reflected in responses to question 12 on the standard evaluation (``This course offered useful learning tools''), which is a key data point.  I focus on this data point because I am being asked by my department to teach in an activities based style with modules like PI-modules, different from the traditional lecture.  The purpose of the activities and group exercises is to satisfy the focus on \textbf{improvement of analysis skill}.  The PI, JITT, and PhET modules are constructed to improve analysis skill through conceptual understanding.  However, upon reflecting on the students' constructive comments, it seems that these modules benefit some students but not all. \\ \hspace{0.1cm}

A vital teaching method emerged in PHYS180, which the students call ``board problems'' in their written responses to evaluations.  It started with an interesting compromise between my desire to move forward in the book faster, and the students' desire to go slowly and have me do examples.  Notice that in the PHYS180 written responses, the students still express quite often a desire for worked examples in class.  In light of all the research-based teaching methods that encourage students to learn through interactions with each other, I decided to have them \textit{work example problems for each other}.  I began by giving an example problem to the class.  The problem would be difficult, and I would either design it myself, or draw it from the current chapter.  Students would then work the problem in groups of 3-4 on the whiteboards, next to other groups.  The class responded positively to this method, and it is reflected in their written responses.  Some even state explicitly that my teaching improved!  The board-problem method works for two reasons: it allows struggling students to see how harder problems are approached by peers, and struggling groups see other groups' strategies and therefore learn from the whole class.  \\ \hspace{0.1cm}

The success of the ``board problems'' technique also reflected the fact that some of the issues PHYS150/PHYS180 students shared were similar to those in PHYS135A/B.  Some students wrote that the pace was too fast, while others reminded me that this was the first time they had encountered mathematical concepts like vector fields.  Students also wrote that class time should be used more effectively, with fewer PhET simulations and more concrete traditional lecturing with examples.  Finally, some students wrote that they didn't benefit from the summarization of scientific articles (meant to practice scientific oral communication).  A handful of students seemed to express the opposite opinion, that there should be more of that activity.  Going forward, I have learned that inclusion of communication activities should be gauged mid-semester, and I will include them if the students are eager and if time permits.

Having reflected extensively on all of the student feedback, I have decided on \textbf{three concrete improvements} to my introductory courses.  In consultation with my department chair, and in studying past PEGP documents in my department, the first improvement will be an increase in traditional lecture content.  The reason is that if every single concept and number in physics is confusing to a first-time student, then merely updating the teaching style with researched-based modules will not help that student.  The traditional lecture style offers the benefit that students see many example problems done in explicit detail, such that they can copy and repeat the technique.  I was taught to never expect this as an undergraduate student.  My colleagues in my department have reassured me that it is necessary to give inexperienced students an explicit starting point.  Thus, going forward in my introductory courses, \textit{a significant fraction of class-time will be spent on concrete examples in the traditional style.} \\ \hspace{0.1cm}

The second major change I will be making to my introductory course teaching style is to slow the pace.  In reading students' remarks, this is the second most common desire on their part.  I was taught at the undergraduate and graduate levels at high speed, with intense focus on both content and mathematical detail.  Of course I must make adjustments for the environment at Whittier College, and not merely teach to myself.  I must \textit{teach to the middle}, as one of my colleagues recommended.  The students felt relief when I began assigning them board-problems, precisely because it allowed them to slow down, and check their work with each other and other groups.  Thus, the addition of board-problems solved both the problem of pace and example problems at once.  The students got a chance to lecture to each other momentarily.  In the coming semesters, \textit{I will include the group-board technique regularly}. One minor adjustment to this technique is that the courses are getting larger enrollments, and we may run out of whiteboard space.  My plan is to sketch the problem, and then allow student groups to design specific examples meant to be exchanged with another group.  This is working in my Fall 2018 PHYS135A sections when it's not feasible to do board-problems.  \\ \hspace{0.1cm}

The third change I'd like to make is to include more applications of calculus in the \textit{calculus-based} introductory sequences.  In my view, more applications of calculus should be included in PHYS150/PHYS180.  From the feedback from my department, I need to include more laboratory activities in PHYS180.  Thus, I propose solving both problems simultaneously.  When I teach \textit{calculus-based} introductory courses in the future, I will use the laboratory activities as a venue for demonstrating the difference between results obtained with and without calculus.  The inclusion of more lab activities is mandatory (and now possible because I'm fully trained on all the equipment).  Thus including calculus concepts in the labs will require little extra effort.  Finally, homework problems involving calculus will be selected from the book's less-difficult category, easing the transition from math to physics contexts. \\ \hspace{0.1cm}

In consultation with my department, I have been focusing on question 17 (``The professor used class time effectively and demonstrated preparation for class.'').  My colleagues believe that many numbers will rise in correlation with question 17.  Struggling students who desired traditional lectures with examples and worksheets likely felt class was not organized because they were unaccustomed to research-based modules like PI or PhET.  Of course I prepared for my courses; I have built an interactive, open-source GB-scale database of lecture content \footnote{see my account on Github.com: \url{https://github.com/918particle/AlgebraBasedMechanics1} or \url{https://github.com/918particle/AlgebraBasedMechanics1}.}.  Going forward, I can use the discussion period during PI modules to focus on helping struggling students one-on-one with a mini-lecture at their table.  The group board problems also afford me the chance to do this.  Finally, providing more traditional lecture content should help the situation. For further analysis of the data in Tab. \ref{tab:courses:intro_eval_1}-\ref{tab:courses:intro_eval_4}, see Appendix A. \\ \hspace{0.1cm}

\end{document}